\hide{
\chapter{Peer Discovery and Peer Selection}
\wip{
  Peer discovery and peer selection are relatively independent for the core components of a node.
  In a first iteration, it may be enough to specify what data is going from
  the peer selection algorithm to
  the core components and what data is going from the core components to the peer selection.
  Also peer discovery and peer selection are relatively independent from each other.
  }

\chapter{Infrastructure}
\label{infrastructure}
\wip{
  WIP: Specific assumptions about the infrastructure that are relevant for the discussion.
}

\section{Internet}
\section{Network Topology}
\section{DNS,NTP}
\section{Topographical distribution of block creating nodes}
\section{TCP}
\section{IP}
MTU
\section{Operating Systems}
\section{Firewall}
\section{Nodes and Hosting}
}

\hide{
\chapter{Haskell}
While the network protocol itself can be implemented in many programming languages,
it has been developed in parallel with a Haskell reference implementation.
In addition to the language agnostic protocol description in the other parts of this document,
this section discusses key aspects of the Haskell implementation.
This section is most useful for people who work with the Haskell reference implementation and
may give some extra insights for anybody who is interested in implementing the
network component.
For understanding the protocol, it is save to skip this section.
\section{Constant Memory Consumption}
\section{The State Machine Framework}
\label{Haskell-state-machine}
}

\hide{
\chapter{Discussion}
\section{Related Work}
\subsection{Other Crypto Currencies}
\subsubsection{PoW Systems}
\subsubsection{PoS Systems}
\subsection{Generic Peer to Peer Systems}
\subsection{Formal Correctness}
}
\wip{
  The correctness of distributed and concurrent systems has been studied intensively for decades.
\begin{description}
\item [Safety properties]
  Prove that a bad thing will never happen.
  \begin{itemize}
  \item Coins cannot be stolen
  \item Preservation of Money
  \item Nodes will not run out of Memory
  \item (Property: Current state is valid) will always hold / never fail
  \end{itemize}
\item[Liveness properties]
  Prove that a desired event will happen.
  \begin{itemize}
  \item Message will be delivered
  \item Consensus will be reached
  \item Transaction will be confirmed
  \item Fairness : the desired event will happen in time. One does not have to wait forever
  \item Starvation
  \item Deadlocks
  \end{itemize}
\item[Temporal logic]
  Tailor made logic for analysing concurrent systems.
  \begin{itemize}
  \item Argue about the temporal order of events in transition systems.
  \item Express safety properties.
  \item Express liveness properties.
  \item Express Fairness.
  \item Prove with model checkers.
  \item Refinement properties.
  \item CTL computation tree logic (safety)
  \item LTL linear time logic (fairness)
  \end{itemize}
\item[Time]
  How does a concurrent system deal with time ?
  \begin{itemize}
  \item Physical clocks / Wall clock time
  \item Logical clocks / Vector clocks / order of events
  \item Order of events : Before , Concurrent, After
  \item Hybrid approaches, Ouroboros, slot-times
  \end{itemize}
\item[Session Types]
     Model protocols and transition systems in a type system.
\item[Pi-calculus]
\item[Process algebras]
\end{description}
}
\hide{
\wip{WIP: Poldercast,etc}
\section{Overview}
}
\section{Design Discussion}
\subsubsection{Why distinguish between node to node and node-to-consumer IPC}
\label{why_distinguish_protocols}
We use two different sets of protocols for these two use cases.

\begin{description}
\item[node-to-node] IPC between nodes that are engaged in the high level Ouroboros
      blockchain consensus protocol.
\item[node-to-consumer] IPC between a Cardano node and a `chain consumer' component such as a
      wallet, explorer or other custom application.
\end{description}

This section describes the differences between those two variants of IPC and why they use
different protocols.

The node-to-node protocol is conducted in a P2P environment
with very limited trust between peers. The node-to-node protocol utilises
store-and-forward over selected \emph{bearers} which form the underlying
connectivity graph. A concern in this setting is asymmetric resource
consumption attacks. Ease of implementation is desirable, but is
subordinate to the other hard constraints.

A node-to-consumer protocol is intended to support blockchain applications
like wallets and explorers, or Cardano-specific caches or proxies. The setting
here is that a consumer trusts a node (a `chain producer') and just wants to
catch up and keep up with the blockchain of that producer. It is assumed that
a consumer only consumes from one producer (or one of a related set of
producers), so unlike in the node-to-node protocol there is no need to choose
between different available chains. The producer may still not fully trust the
consumer and does not want to be subject to asymmetric resource
consumption attacks. In this use case, because of the wider range of
applications that wish to consume the blockchain, having some options that are
easy to implement is more important, even if this involves a trade-off with
performance. That said, there are also use cases where tight integration is
possible and making the most efficient use of resources is more desirable.

There are a number of applications that simply want to consume the blockchain,
but are able to rely on an upstream trusted or semi-trusted Cardano consensus
node. These applications do not need to engage in the full consensus protocol,
and may be happy to delegate the necessary chain validation.

Examples include 3rd party applications that want to observe the blockchain,
examples being business processes triggered by transactions or analytics.  It
may also include certain kinds of light client that wish to follow the
blockchain but not do full validation.

Once one considers a node-to-consumer protocol as a first class citizen then it
opens up opportunities for different system architecture choices.
The architecture of the original Cardano Mainnet release was entirely homogeneous:
every node behaved the same, each trusted nothing but itself and paid the full
networking and processing cost of engaging in the consensus protocol.  In
particular everything was integrated into a single process: the consensus
algorithm itself, serving data to other peers and components such as the wallet
or explorer. If we were to have a robust and efficient node-to-consumer protocol
then we can make many other choices.

With an efficient \emph{local} IPC protocol we can have applications
like wallets and explorers as separate processes. Even for tightly
integrated components it can make sense to run them in separate OS
processes and using associated OS management tools. Not only are the
timing constraints for a consensus node much easier to manage when
it does not have to share CPU resources with chain consumers,
but it enables sophisticated end-users to use operating system features
to have finer control over resource consumption.
There have been cases in production where a highly loaded wallet component takes
more than its allowed allocation of CPU resources and causes the local
node to miss its deadlines.  By giving a consensus node a dedicated
CPU core it becomes easier to provide the necessary hard real
time guarantees. In addition, scaling on multi-core machines is
significantly easier with multiple OS processes than with a
multi-threaded OS process with a shared-heap. This could allow
larger capacity Cardano relay deployments where there are multiple
network facing proxy processes that all get their chain from a single
local consensus node.

With an efficient \emph{network} IPC protocol we can do similar things
but extend it across multiple machines. This permits: large
organisations to achieve better alignment with their security
policies; clusters of relays operated by a single organisation to use
the more efficient (less resource costly) node-to-consumer protocol
instead of the node-to-node protocol; and wallet
or explorer-like applications that need to scale out, and are able to
make use of a trusted node.

\hide{
\section{Requirements}
\section{Threat Vectors}
\wip{
\begin{description}
\item [Generic Attacks against IP networks]
\item [Attacks against a specific implementation of the protocol]
\item [Attacks against a specific configuration of the system]
\item [Attacks against the network protocol itself]
\item [Attacks against Ouroboros]
\item [Clever combinations of the above]
\end{description}
}
\subsubsection{Asymptotic Resource Consumption}
\section{Results from Simulations}
\section{Pub Sub}
\section{Of the Shelf Protocols}
\section{Congestion Control}
\subsection{DeltaQ and Back-pressure}
\label{deltaq-discussion}
\wip{WIP: discuss DeltaQ and Back-pressure}

\section{Meta Requirements}
\subparagraph{Work in Progress}
This document is evolved in parallel with the work on the protocol design and
the reference implementation.

\subparagraph{The Document should be Comprehensive}
\begin{itemize}
\item Top down approach.
\item Provide the big picture.
\item Usable as a reference point for a broader discussion.
\item Cover every aspect that is related to network connections.
\item Every aspect should at least have a place in the table of contents.
  If there are holes and parts that are not covered the document should say what is missing.
\item Stand alone readable with links to where missing pieces can be found.
\end{itemize}

\subparagraph{Detailed}
\begin{itemize}
\item Sufficient details to allow for new independent implementations that are compatible with
the reference implementation
\item Language agnostic (it is save to skip the Haskell specific parts)
\item Design discussions
\end{itemize}
\subparagraph{Structured}
\begin{itemize}
\item Parts of the document should be in a logical connection
\end{itemize}
\subparagraph{Workflow}
}

\appendix
\chapter{CDDL Specification of the Protocol Messages}
\label{CBOR-section}
\hsref{ouroboros-network/src/Ouroboros/Network/Protocol/PingPong/Codec.hs}
\label{included-cddl}
This Sections contains the CDDL\cite{cddl} specification
of the binary serialisation format of the network protocol messages.

To keep this Section in close sync with the actual Haskell implementation
the names of the Haskell identifiers have been reused for the corresponding
CBOR types (with the first letter converted to lower case).
Note, that, for readability, the previous Sections used simplified message identifiers,
for example {\tt RequestNext} instead of {\tt msgRequestNext}, etc.
Both identifiers refer to the same message format.

All transmitted messages satisfy the shown CDDL specification.
However, CDDL, by design, also permits variants in the encoding that are not valid in the protocol.
In particular, the notation ${\tt [} ... {\tt ]}$ in CDDL can be used for both fixed-length
and variable-length CBOR-list, while only one of the two encodings is valid in the protocol.
We add comments in specification to make clear which encoding must be used.

Note that, in the case of the request-response mini protocol (Section~ref{request-response-protocol})
there in only ever one possible kind of message in each state.
This means that there is no need to tag messages at all
and the protocol can directly transmit the plain request and response data.

\wip{TODO: test that haskell(message) => cddl(message) }
\lstinputlisting{messages.cddl}
\bibliographystyle{apalike}
\bibliography{references}

\hide{
\chapter{Key Figures of the Protocol and the P2P Network}
This section list some key figures of the network protocol.
There is a variety of figures that quantify some aspects of the protocol, for example:
\begin{itemize}
\item Configuration parameters that are explicitly set in the protocol.
\item Requirements and performance targets.
\item Implicit assumptions about about network bandwidths, etc.
\item Estimates from simulations and game theoretic results.
\end{itemize}
These figures can be a fixed value, a possible interval, a distribution of values,
or just a rough estimate and typically these figures depend on each other.
The figures in this section are not set in stone, but they should help to give a baseline that helps
to understand the protocol design.

Block chain parameters:\\
\begin{tabular}{p{4cm}p{1cm}p{6cm}p{1cm}} \hline
  maximum block size   & = & 2 M Bytes                                    &  \\ \hline
  slot time            & = & 20s                                          &  \\ \hline
  epoch length         & = & 22600 slots $\simeq$ 5 days                  &  \\ \hline
  intrinsic probability of a fork & = &                                   &  \\ \hline
  K parameter (maximal roll back) & = &                                   &  \\ \hline
\end{tabular}\\

Transaction:\\
\begin{tabular}{p{4cm}p{1cm}p{6cm}p{1cm}} \hline
  size of a transaction                & = &  some K Bytes               &  \\ \hline
  through-put transactions per second   & = &  15                        &  \\ \hline
\end{tabular}\\

Network topology:\\
\begin{tabular}{p{4cm}p{1cm}p{6cm}p{1cm}} \hline
  maximum hops                         & $\le$ &  5                         &  \\ \hline
  maximum number of neighbours          & = &  5                         &  \\ \hline
\end{tabular}

\begin{tabular}{p{4cm}p{1cm}p{6cm}p{1cm}} \hline
  bandwidth stake pool                             & = &  > 100Mbit/s             &  \\ \hline
  bandwidth small stake holder                    & = &  ~ 100Mbit/s             &  \\ \hline
  bandwidth none staking chain consumer          & = &  5                       &  \\ \hline
  latency  between stake pool nodes                & <10 ms                       &  \\ \hline
\end{tabular}\\


\chapter{Nomenclature}
\begin{description}
\item[Adversary / Adversarial Action]
  acting in way to subvert the correct (or performant) operation of the distributed protocol.
  Note that non-performance of certain functions at appropriate times can
  fall into this category.
\item[Core DIF]
  The set of end points that belong to the (major)
  stakepools; (the term DIF taken from RINA\ref{RINA} where it denotes
  a (potentially closed) set of potential participants.
\item[egress,ingress]
\item[head of line blocking]
\end{description}
}
